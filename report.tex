\documentclass[12pt,a4paper]{article}
\input{latexmacros.tex}

\title{Using AC-3 to come up with a good title}
\author{David Hildering \and Dennis Lindberg \and Joel Maxson \and Helen Sand \and Andy S. Tatman}
\date{\today}
\hypersetup{pdfauthor={David Hildering \and Dennis Lindberg \and Joel Maxson \and Helen Sand \and Andy S. Tatman}, pdftitle={Using AC-3 to come up with a good title}}

\begin{document}

\maketitle

\begin{abstract}
We give a toy example of a report in \emph{literate programming} style.
The main advantage of this is that source code and documentation can
be written and presented next to each other.
We use the listings package to typeset Haskell source code nicely.
\end{abstract}

\vfill

\tableofcontents

\clearpage

% We include one file for each section. The ones containing code should
% be called something.lhs and also mentioned in the .cabal file.

% \input{Howto.tex}

\section{Introduction}
TODO: What is AC-3, why use it (etc)...

AC-3 paper: \cite{AC3}

TODO: Add a note that AC-3 cannot \emph{solve} -> See Dennis' example, now in ac3Tests.lhs

\section{The skeleton files}
These files form the basis of our implementation, which we then use to solve various problems.

\input{lib/AC3Solver.lhs}

\input{lib/Backtracking.lhs}

\section{The problem files} % TODO: Maybe a better name for this?

\input{lib/NQueens.lhs}

\input{lib/GraphCol.lhs}

\input{lib/Scheduling.lhs}

\input{lib/Sudoku.lhs}

% \section{The Main file} %already in the main file

\input{exec/Main.lhs}

\section{The test file(s)}

\input{test/ac3Tests.lhs}


\section{Conclusion \& Future work}\label{sec:Conclusion}

Finally, we can see that \cite{liuWang2013:agentTypesHLPE} is a nice paper.


\addcontentsline{toc}{section}{Bibliography}
\bibliographystyle{alpha}
\bibliography{references.bib}

\end{document}
