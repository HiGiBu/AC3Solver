\documentclass[12pt,a4paper]{article}
\input{latexmacros.tex}

\title{General purpose AC-3 solver with various applications}
\author{David Hildering \and Dennis Lindberg \and Joel Maxson \and Helen Sand \and Andy S. Tatman}
\date{\today}
\hypersetup{pdfauthor={David Hildering, Dennis Lindberg, Joel Maxson, Helen Sand, Andy S. Tatman}, pdftitle={General purpose AC-3 solver with various applications}}

\begin{document}

\maketitle

\begin{abstract}
We give a toy example of a report in \emph{literate programming} style.
The main advantage of this is that source code and documentation can
be written and presented next to each other.
We use the listings package to typeset Haskell source code nicely.
\end{abstract}

\vfill

\tableofcontents

\clearpage

% We include one file for each section. The ones containing code should
% be called something.lhs and also mentioned in the .cabal file.

% \input{Howto.tex}

\section{Introduction}
% DONE: What is AC-3, why use it (etc)...
Many problems in computer science can be written as a set of possible values for each 
body in the problem instance, and a set of restrictions, or constraints, between pairs of bodies.
Using this method, we can then define a problem instance as a graph, where each vertex has a set of possible values, or domain,
and where each edge between two vertices is a constraint.  

This forms the basis behind the arc consistency algorithms discussed in~\cite{AC3}, including the AC-3 algorithm 
we use in our implementation. 
The AC-3 algorithm aims to efficiently iterate over all of the constraints, 
and remove any values for which the constraints is invalidated.

Take as an example $X$ with domain $[1,2]$ and $Y$ with domain $[2,3]$, with an edge connecting them representing the contraint \verb:(==):.
The AC-3 algorithm will remove the value $1$ from $X$'s domain and the value $3$ from $Y$'s domain, 
as there is no value in the other body's domain such that the constraint is true for $X=1$ or $Y=3$.

% AC-3 paper: \cite{AC3}

% DONE: Add a note that AC-3 cannot \emph{solve} -> See Dennis' example, now in ac3Tests.lhs
% TODO: Add a visual graph?
Notably, AC-3 cannot \emph{solve} problems. As an example, take 3 entities $X,Y,Z$, 
each with domain $[0,1]$, and with edges \verb:(/=): between each. 
The AC-3 algorithm does not purge any values. For each value, the other entities have a 
value for which the constraint holds.
However, if we try to find a solution with backtracking, we find that no solution exists, as 
the three entities need to all have a unique value. 

As such, AC-3's main purpose is to reduce the search space, by pruning values for which no
 solution can exist. If at least one vertex has no possible values, then there is no solution.
Else, we can use backtracking to determine if a solution exists, and if so what said solution is.

We discuss our AC-3 and backtracking implementation in Section~\ref{sec:skeleton}. 
We then introduce a number of different computer science problems in Section~\ref{sec:problems}, and show how our implementation
attempts to solve them. 

\section{The skeleton files}\label{sec:skeleton}
These files form the basis of our implementation, which we then use to solve various problems.

\input{lib/AC3Solver.lhs}

\input{lib/Backtracking.lhs}

\section{The problem files}\label{sec:problems} % TODO: Maybe a better name for this?

\input{lib/NQueens.lhs}

\input{lib/GraphCol.lhs}

\input{lib/Scheduling.lhs}

\input{lib/Sudoku.lhs}

% \section{The Main file} %already in the main file

\input{exec/Main.lhs}

% \section{The test file(s)}

% \input{test/ac3Tests.lhs}


\section{Conclusion \& Future work}\label{sec:Conclusion}

Finally, we can see that \cite{liuWang2013:agentTypesHLPE} is a nice paper.

% \section{Future works??} %moved into conclusion

\addcontentsline{toc}{section}{Bibliography}
\bibliographystyle{alpha}
\bibliography{references.bib}

\end{document}
